\documentclass[10pt]{beamer}

%%%%%%%%%%%%%%%%%%%%%%%%%
%   Options for notes   %
%%%%%%%%%%%%%%%%%%%%%%%%%
% \setbeamertemplate{note page}[plain]
\setbeameroption{show notes}
% \setbeameroption{show notes on second screen}
% \setbeameroption{show only notes}
% \setbeameroption{show only slides with notes}

%%%%%%%%%%%%%%%%%%%%%%%%%
%    useful packages    %
%%%%%%%%%%%%%%%%%%%%%%%%%
% \usepackage{ctex, hyperref}
\usepackage{hyperref}
\usepackage[T1]{fontenc}
\usepackage[utf8]{inputenc}
\usepackage{framed}
\usepackage{multicol}
\usepackage{ragged2e}
\justifying\let\raggedright\justifying
% other packages
\usepackage[export]{adjustbox}
\usepackage{latexsym,booktabs,calligra,xcolor}
\usepackage{graphicx,pstricks,listings,stackengine}
\usepackage{makecell}
\usepackage{CUHK}
\usepackage{algorithm,algpseudocode}


\usepackage{amsmath,amsfonts,mathtools,amsthm,amssymb}
\usepackage{pifont}
\usepackage{bm,ulem}
\usepackage{empheq}
\newcommand*\widefbox[1]{\fbox{\hspace{2em}#1\hspace{2em}}}
\usepackage[font=scriptsize]{caption}
\usepackage{advdate}
\usepackage{extarrows}
\usepackage{subcaption}
\usepackage{bbm}
% \usepackage[superscript,biblabel]{cite}
\usepackage[super,square]{natbib}

\let\implies\Rightarrow
\let\impliedby\Leftarrow
\let\iff\Leftrightarrow
\let\epsilon\varepsilon
\newcommand{\cmark}{\ding{51}}%
\newcommand{\xmark}{\ding{55}}%


% KL divergence
\DeclarePairedDelimiterX{\infdivx}[2]{(}{)}{%
  #1\;\delimsize\|\;#2%
}
\newcommand{\infdiv}{D_{\mathbb{KL}}\infdivx}
\DeclarePairedDelimiter{\norm}{\lVert}{\rVert}

% argmin max
\DeclareMathOperator*{\argmax}{\arg\!max}
\DeclareMathOperator*{\argmin}{\arg\!min}
\DeclareMathOperator*{\mean}{mean}
\DeclareMathOperator*{\vperp}{\text{\rotatebox{90}{$\models$}}}
\DeclareMathOperator*{\diag}{diag}



%%%%%%%%%%%%%%%%%%%%%%%%%
% user-defined commands %
%%%%%%%%%%%%%%%%%%%%%%%%%
% \renewenvironment{cases}[1][l]{\matrix@check\cases\env@cases{#1}}{\endarray\right.}
\def\cmd#1{\texttt{\color{red}\footnotesize $\backslash$#1}}
\def\env#1{\texttt{\color{blue}\footnotesize #1}}
\def\ulbf#1{\textbf{\underline{#1}}}
\def\todo#1{\large\color{red}\textit{#1}}
\xdefinecolor{cuhksz1}{rgb}{0.457,0.0585,0.42578}      %RGB(117,15,109)
\xdefinecolor{cuhksz2}{rgb}{0.86328,0.63671875,0}      %RGB(221,163,0)
\xdefinecolor{cuhksz3}{rgb}{0.953125,0.87109,0.6875}   %RGB(244,223,176)


\definecolor{deepred}{rgb}{0.6,0,0}
\definecolor{frenchblue}{rgb}{0.0, 0.25, 0.63}

\lstset{
    basicstyle=\ttfamily\small,
    keywordstyle=\bfseries\color{cuhksz1},
    emphstyle=\ttfamily\color{deepred},    % Custom highlighting style
    stringstyle=\color{cuhksz2},
    numbers=left,
    numberstyle=\small\color{cuhksz3},
    rulesepcolor=\color{red!20!green!20!blue!20},
    frame=shadowbox,
}

\newcommand{\advanceday}[1][14]{
\begingroup
\AdvanceDate[#1]
\today
\endgroup
}
\algnewcommand\algorithmicinitialize{\textbf{initialize: }}
\algnewcommand{\Initialize}[1]{\State \algorithmicinitialize #1}


%%%%%%%%%%%%%%%%%%%%%%%%%
%     presenter info    %
%%%%%%%%%%%%%%%%%%%%%%%%%
\author{
Your Name
}
\title{
Template Usage Examples
}
\subtitle{
A demonstration of CUHK Beamer template features
}
\institute[Short Name]{
Your Institute Name
}
\date{\today}

\begin{document}

\begin{frame}
    \titlepage
    \vfill
    \begin{figure}
        \begin{center}
            \includegraphics[width=0.3\linewidth]{pic/STAT_SFO_CU_logo_en_chi_small.png}
        \end{center}
    \end{figure}
\end{frame}

\section{Template Features}

\begin{frame}{1. Speaker Notes}
    This slide demonstrates how to use speaker notes.
    
    \begin{itemize}
        \item Regular content appears on the presentation
        \item Notes appear only in the notes view
    \end{itemize}
\end{frame}

\note[enumerate]{
    \item This is a speaker note
    \item Notes can be used for additional information
    \item Enable notes by uncommenting in the preamble
}

\begin{frame}{2. Mathematical Notation}
    Rich mathematical support:
    \begin{align}
        \bm{P}^*=\argmin_{\bm{P}\in\mathbb{R}_+^{m\times n}} & ~ \langle\bm{P},\bm{C}\rangle_\mathsf{F} \\
        \text{s.t.} & ~ \bm{P1}_n = \bm{s},~ \bm{P}^\mathsf{T}\bm{1}_m = \bm{d}
    \end{align}
    
    Common operators: $\argmin$, $\argmax$, $\infdiv{\bm{P}}{\bm{Q}}$
\end{frame}

\note[enumerate]{
    \item Mathematical notation is typeset using amsmath
    \item Special operators are predefined in preamble.tex
}

\begin{frame}{3. Bilingual Support}
    \begin{itemize}
        \item English content example
        \item 中文内容示例
        \item Mixed content example
    \end{itemize}
    
    \begin{center}
        \large
        \boxed{\makecell{Multi-line\\Content}}
    \end{center}
\end{frame}

\note[enumerate]{
    \item Template supports both Chinese and English
    \item Use fontspec package for Chinese support
}

\begin{frame}{4. CUHK Colors}
    University color scheme:
    \begin{itemize}
        \item \textcolor{cuhk1}{Primary Color (cuhk1)}
        \item \textcolor{cuhk2}{Secondary Color (cuhk2)}
        \item \textcolor{cuhk3}{Tertiary Color (cuhk3)}
    \end{itemize}
    
    \begin{center}
        \boxed{\textcolor{cuhk1}{Colored Boxed Text}}
    \end{center}
\end{frame}

\note[enumerate]{
    \item Colors are defined in CUHK.sty
    \item Use \textbackslash textcolor\{cuhk1\}\{text\} for coloring
}

\begin{frame}{5. Common Environments}
    \begin{columns}
        \column{0.5\textwidth}
        Boxed text:
        \begin{center}
            \boxed{\makecell{Multi-line\\boxed content}}
        \end{center}
        
        \column{0.5\textwidth}
        Figure example:
        \begin{figure}
            \centering
            % \includegraphics[width=0.8\textwidth]{pic/example.pdf}
            \caption{Figure caption}
        \end{figure}
    \end{columns}
\end{frame}

\note[enumerate]{
    \item Use \textbackslash boxed for mathematical content
    \item Use makecell for multi-line content
    \item Figures can be included with \textbackslash includegraphics
}

\begin{frame}{6. Bibliography}
    \begin{itemize}
        \item Citation example\cite{example2023}
        \item Multiple citations\cite{example2023,example2024}
    \end{itemize}
    
    \begin{center}
        \large
        \boxed{Add references in ref.bib}
    \end{center}
\end{frame}

\note[enumerate]{
    \item Use \textbackslash cite\{key\} for citations
    \item Add entries in ref.bib
    \item Run bibtex for bibliography
}

\begin{frame}{7. Algorithm}
    \begin{algorithm}[H]
        \begin{algorithmic}[1]
            \State \textbf{Input:} $\bm{P}, \bm{Q}$
            \State \textbf{Output:} $\bm{P}^*$
            \State $\bm{P}^* \leftarrow \bm{P}$
            \State $\bm{P}^* \leftarrow \bm{Q}$
        \end{algorithmic}
        \caption{Algorithm caption}
        \label{alg:algorithm}
    \end{algorithm}
\end{frame}

\note[enumerate]{
    \item Use \textbackslash begin\{algorithm\}\[H\] for algorithm
    \item Use \textbackslash begin\{algorithmic\}\[1\] for algorithm
    \item Use \textbackslash State for algorithm steps
}

\begin{frame}{8. Table}
    \begin{table}[H]
        \centering
        \begin{tabular}{ccc}
            \toprule
            Header 1 & Header 2 & Header 3 \\
            \midrule
            Data 1 & Data 2 & Data 3 \\
            \bottomrule
        \end{tabular}
        \caption{Table caption}
        \label{tab:table}
    \end{table}
\end{frame}

\note[enumerate]{
    \item Use \textbackslash begin\{table\}\[H\] for table
    \item Use \textbackslash begin\{tabular\}\{ccc\} for table
    \item Use \textbackslash toprule, \textbackslash midrule, and \textbackslash bottomrule for table borders
}



\begin{frame}[allowframebreaks]{References}
    \bibliographystyle{plain}
    \bibliography{ref}
\end{frame}

\note[enumerate]{
    \item Use \textbackslash bibliographystyle\{plain\} for bibliography
    \item Use \textbackslash bibliography\{ref\} for references
}
\end{document}